%-------------------------------------------------------------------------------
%	SECTION TITLE
%-------------------------------------------------------------------------------
\cvsection{Research}
%	CONTENT
%-------------------------------------------------------------------------------
\begin{cventries}

%---------------------------------------------------------
  \cventry
    {Chrispen Mafirabadza, Trevor Makausi, Pallavi Khatri} % 
    {\href{https://dl.acm.org/citation.cfm?id=2979831}{Efficient Power Aware AODV Routing Protocol in MANET}\articleLink} % 
    {Bikaner, India (AITC - ACM)} % Location
    {12 -13 August, 2016} % Date(s)
    {
      \begin{cvitems} % Description(s) of experience/contributions/knowledge
        \item {The research was conducted for AICTC 2016 Proceedings of the International Conference on Advances in Information Communication Technology and Computing under the theme of Ad-hoc Networks.The researched carried out was a proposal for an Efficient Power Aware Ad-hoc on-Demand Distance Vector(EPAAODV) protocol. This is a modification of the normal functioning of the Ad-hoc on-Demand Distance Vector protocol. The proposed technique made an attempt to increase network lifetime of a MANET. An energy consumption analysis was performed on the proposed work and the existing protocol. Other parameters such as, packet delivery ratio (PDR) and throughput
        were also analyzed using network simulator (NS2).} 
      \end{cvitems}
    }

%---------------------------------------------------------
  \cventry
    {Tendai Marengereke, Trevor Makausi, David Fadaraliki, Walter Mambodza} % Affiliation/role
    { Open Source Software for Universities} % Organization/group
    {Zanzibar, Tanzania} % Location
    {22-23 November, 2018} % Date(s)
    {
      \begin{cvitems} % Description(s) of experience/contributions/knowledge
        \item {The research was carried out for the Ububtu-Net Connect 2018 annual conference of Ubuntu Net Alliance, the Regional Research and Education Network for Eastern and Southern Africa. The theme was under SMART Governance:Services and Tools. The research used a systematic approach to review the available open source technologies that can be used in Universities to enhance their service as well as reduce overhead cost.}
      \end{cvitems}
    }


%---------------------------------------------------------
\end{cventries}
